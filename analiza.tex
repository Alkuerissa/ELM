\documentclass{article}
\usepackage[hidelinks]{hyperref}

\usepackage{polski}
\usepackage[utf8]{inputenc}
\usepackage{indentfirst}
\usepackage{float}
\title{Projekt aplikacji do ekstremalnego uczenia maszynowego do klasyfikacji big data}
\author{Abdelkarim Ahmed, Hernik Aleksandra}
\begin{document}
Wydział Matematyki i Nauk Informacyjnych Politechniki Warszawskiej
\vspace*{\fill}
\begin{center}
\begin{minipage}{.9\textwidth}
\maketitle
\begin{center}Wersja 1.2\end{center}
\end{minipage}
\end{center}
\vfill % equivalent to \vspace{\fill}
\clearpage
\noindent
\begin{table}[H]
\begin{tabular}{|l|l|l|r|}
\hline
\textbf{Data} & \textbf{Autor} & \textbf{Opis zmian} & \textbf{Wersja} \\
\hline
10.10.2016 & Ahmed Abdelkarim & Diagram przypadków użycia & 0.1 \\
10.10.2016 & Aleksandra Hernik & Wstępna wersja harmonogramu i wymagań & 0.2 \\
19.10.2016 & Aleksandra Hernik & Korekty harmonogramu i wymagań & 0.3 \\
02.11.2016 & Ahmed Abdelkarim & Stworzenie i wprowadzenie szablonu & 1.0 \\
02.11.2016 & Ahmed Abdelkarim & Opis przypadków użycia & 1.1 \\
02.11.2016 & Aleksandra Hernik & Opis biznesowy, dokończenie harmonogramu & 1.2 \\
\hline
\end{tabular}
\end{table}
\tableofcontents
\clearpage
\section{Specyfikacja}
\subsection{Opis biznesowy}
\subsection{Wymagania funkcjonalne}
\begin{itemize}
\item Przygotowanie danych trenujących i testowych w formie plików csv
\item Trenowanie sieci neuronowej o architekturze ELM
\item Klasyfikacja danych testowych przez wytrenowaną sieć
\item Porównywanie wyników klasyfikacji do spodziewanych rezultatów
\item Analiza jakości klasyfikacji i  czasu trwania treningu
\item Interfejs graficzny
\end{itemize}
\subsection{Wymagania niefunkcjonalne}
\begin{itemize}
\item Działanie na komputerach wydziałowych
\item Działanie o kilka rzędów wielkości szybsze od tradycyjnych metod uczenia
\item Przejrzysty interfejs graficzny
\item Czytelne wykresy
\end{itemize}
\subsection{Harmonogram projektu}
\begin{table}[H]
\begin{tabular}{|l|r|r|r|}
\hline
\textbf{Etap} & \textbf{Czas} & \textbf{Początek} & \textbf{Koniec} \\
\hline
Przygotowanie pierwszego rozdziału dokumentacji & 24 dni & 10.10.2016 & 02.11.2016 \\
Opracowanie modelu pracy sieci ELM & 14 dni & 17.10.2016 & 30.10.2016 \\
Analiza narzędzi & 7 dni & 31.10.2016 & 06.11.2016 \\
Implementacja i napisanie drugiego rozdziału pracy & 14 dni & 07.11.2016 & 20.11.2016 \\
Tworzenie i wykonywanie testów dla small data & 7 dni & 21.11.2016 & 27.11.2016 \\
Tworzenie i wykonywanie testów dla big data & 7 dni & 28.11.2016 & 04.12.2016 \\
Opis wniosków z testów & 7 dni & 05.12.2016 & 11.12.2016 \\
\hline
\end{tabular}
\end{table}
\end{document}