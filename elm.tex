\documentclass{article}
\usepackage[hidelinks]{hyperref}
%\usepackage{helvetica} % uses helvetica postscript font (download helvetica.sty)
%\usepackage{newcent}   % uses new century schoolbook postscript font 
%\setlength{\textwidth}{5.5in} % set width of text portion

\usepackage{polski}
%\usepackage[polish, english]{babel}
%\usepackage[T1]{fontenc} % T1, OT4
\usepackage[utf8]{inputenc}
\usepackage{indentfirst}
\usepackage{float}
\begin{document}

\section*{Wprowadzenie}
\section*{Wykaz ważniejszych oznaczeń i akronimów}
\section{Opracowanie modelu pracy sieci do ekstremalnego uczenia maszynowego ELM}
\subsection{Rys historyczny}
\subsection{Założenia dla sieci neuronowych klasy ELM}
\subsection{Wnioski i uwagi}
\section{Wykonanie implementacji ELM w Pythonie i Matlabie}
\subsection{Projekt aplikacji sieci ELM w Pythonie}
\subsection{Projekt aplikacji sieci ELM w Matlabie}
\subsection{Wnioski i uwagi}
\section{Trening ELM dla wybranych benchmarków big data i small data}
\subsection{Opis wybranych benchmarków}
\subsection{Trening dla instancji klasy small data}
\subsection{Trening dla instancji klasy big data}
\subsection{Wnioski i uwagi}
asdfasdfsad
\section{Eksperymenty obejmujące instancje klasyfikacji wraz z badaniem ich wydajności}
\section*{Podsumowanie}
\section*{Bibliografia}
\section*{Wykaz rysunków}
\section*{Wykaz tabel}
\section*{Dodatek. Instrukcja obsługi aplikacji}
\end{document}













