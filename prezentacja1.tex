\documentclass{beamer}
\usepackage{polski}
\usepackage[utf8]{inputenc}
\usepackage{graphics}
\usepackage{minted}
\usepackage{fvextra}
\usepackage{caption}
\usepackage{epigraph}

\definecolor{NiceGreen}{RGB}{40,200,0}
\definecolor{CoolDarkGray}{RGB}{50,50,50}


\title{\texttt{Projekt aplikacji do ekstremalnego uczenia maszynowego do klasyfikacji big data}}
\author{Ahmed Abdelkarim, Aleksandra Hernik}

\begin{document}
\setbeamercolor{background canvas}{bg=CoolDarkGray}
\setbeamercolor{title}{fg=NiceGreen}
\setbeamercolor{frametitle}{fg=NiceGreen}
\setbeamercolor{structure}{fg=white}
\setbeamertemplate{section in toc}{%
    \textcolor{NiceGreen}{\inserttocsectionnumber)}   \inserttocsection \par}

\addtobeamertemplate{navigation symbols}{}{%
    \usebeamerfont{footline}%
    \usebeamercolor[fg]{footline}%
    \hspace{1em}%
    \insertframenumber/\inserttotalframenumber
}

\setbeamercolor{normal text}{fg=white}\usebeamercolor*{normal text}
\begin{frame}
  \maketitle
\end{frame}

\begin{frame}{Plan}
  \tableofcontents[currentsection]
\end{frame}

\section{Czym jest ELM?}
\begin{frame}{1}
\end{frame}
\begin{frame}{1}
\end{frame}
\section{Cel pracy}
\begin{frame}{Cele projektu}
\begin{itemize}
\item Uruchomienie algorytmu na testach różnych rozmiarów
\item Zmierzenie czasu trwania uczenia
\item Przetestowanie jakości uczenia
\item Porównanie aplikacji w Pythonie i Matlabie
\item Rozważenie możliwych ulepszeń algorytmu
\end{itemize}
\end{frame}
\section{Implementacja}
\begin{frame}{Zadania}
\begin{itemize}
\item Obudowanie istniejącego kodu klasami, ułatwiającymi puszczanie testów i zbieranie informacji
\item Uruchomienie testów i analiza wyników
\item[?] Wprowadzenie własnych funkcji jakości 
\item[?] Przetestowanie innych funkcji aktywacji
\end{itemize}
\end{frame}

\begin{frame}{Diagram klas}
\begin{figure}
	\includegraphics[scale=0.26]{classes.png}
\end{figure}
\end{frame}


\begin{frame}{Harmonogram}

\end{frame}


\begin{frame}{Źródła}

\end{frame}

\end{document}

\endinput