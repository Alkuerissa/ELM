\documentclass[pl]{minipw} % wszystkie ustawienia szablonu są w minipw.cls; if in English, change [pl] to [en]
\allowdisplaybreaks
\usepackage{indentfirst}
%\usepackage[hidelinks]{hyperref}
\usepackage[all]{nowidow}
\usepackage{caption}
\usepackage{graphicx}
\usepackage{tabularx}
\usepackage{polski}
\usepackage{bm}
\usepackage{amsfonts}
\usepackage{amsmath}
\usepackage[utf8]{inputenc}
\usepackage{indentfirst}
\usepackage{float}
\usepackage{enumitem}
\usepackage{listings}
\setlength{\parindent}{5mm} % wcięcie akapitowe 5mm, zarządzenie Rektora


\begin{document}
\sloppy

{\selectlanguage{english}
\begin{abstract}
The goal of thesis \textit{Application of Extreme Learning Machines for Big Data Classification} is analysis of relatively new and unexplored architecture of neural network, used in classification problem, with emphasis on possible application in big data problems. \\
In order to achieve it, two applications were created -- one in Python, using a~ready-made library \textit{HP-ELM}, and one in Matlab, featuring our own implementation of ELM. \\
Both programs were used in the following experiments: predicting match result in the game \textit{Dota~2}, based on state at the of the match -- i.e. assigning one of two classes to each game (victory of either team) -- and forest cover type prediction, based on geographic variables. \\
Experiments confirmed the assertion about ELM's high speed (of the order of seconds for 1000 neurons and tens of thousands of samples) and satisfactory effectiveness -- however, the results differed between the applications -- our own implementation was much better at forest cover type prediction. \\
\noindent \textbf{Keywords:} big data, classification, extreme learning machine, neural network, machine learning, Matlab, Python
\end{abstract}}

\end{document}



















