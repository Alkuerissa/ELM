\documentclass[pl]{minipw} % wszystkie ustawienia szablonu są w minipw.cls; if in English, change [pl] to [en]
\allowdisplaybreaks
\usepackage{indentfirst}
%\usepackage[hidelinks]{hyperref}
\usepackage[all]{nowidow}
\usepackage{caption}
\usepackage{graphicx}
\usepackage{tabularx}
\usepackage{polski}
\usepackage{bm}
\usepackage{amsfonts}
\usepackage{amsmath}
\usepackage[utf8]{inputenc}
\usepackage{indentfirst}
\usepackage{float}
\usepackage{enumitem}
\usepackage{listings}
\setlength{\parindent}{5mm} % wcięcie akapitowe 5mm, zarządzenie Rektora



\begin{document}
\sloppy


%\selectlanguage{polish}
\begin{abstract}
\setcounter{page}{1}
Celem pracy "Projekt aplikacji do ekstremalnego uczenia maszynowego do klasyfikacji big data" jest przeanalizowanie stosunkowo nowej i~niezbadanej architektury sieci neuronowej, ze szczególnym uwzględnieniem jej możliwych zastosowań w~problemach big data.
Jest on osiągany poprzez stworzenie dwóch aplikacji -- w~Pythonie, korzystającej z~gotowej biblioteki i~w~Matlabie, będącej własną implementacją algorytmu.\\

\noindent \textbf{Słowa kluczowe:} big data, ekstremalne uczenie maszynowe, klasyfikacja, Matlab, Python, sieci neuronowe, uczenie maszynowe
\end{abstract}


\end{document}



















