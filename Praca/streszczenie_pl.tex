\documentclass[pl]{minipw} % wszystkie ustawienia szablonu są w minipw.cls; if in English, change [pl] to [en]
\allowdisplaybreaks
\usepackage{indentfirst}
%\usepackage[hidelinks]{hyperref}
\usepackage[all]{nowidow}
\usepackage{caption}
\usepackage{graphicx}
\usepackage{tabularx}
\usepackage{polski}
\usepackage{bm}
\usepackage{amsfonts}
\usepackage{amsmath}
\usepackage[utf8]{inputenc}
\usepackage{indentfirst}
\usepackage{float}
\usepackage{enumitem}
\usepackage{listings}
\setlength{\parindent}{5mm} % wcięcie akapitowe 5mm, zarządzenie Rektora



\begin{document}
\sloppy


%\selectlanguage{polish}
\begin{abstract}
\setcounter{page}{1}
Celem pracy \textit{Projekt aplikacji do ekstremalnego uczenia maszynowego do klasyfikacji big data} jest przeanalizowanie stosunkowo nowej i~niezbadanej architektury sieci neuronowej~w problemie klasyfikacji, ze szczególnym uwzględnieniem jej możliwych zastosowań w~problemach big data.
Jest on osiągany poprzez stworzenie dwóch aplikacji -- w~Pythonie, korzystającej z~gotowej biblioteki \textit{HP-ELM} i~w~Matlabie, będącej własną implementacją algorytmu. \\
Obie aplikacje zostały wykorzystane w~wybranych eksperymentach: przewidywaniu wyniku meczu~w grze \textit{Dota 2} na podstawie stanu gry w momencie zakończenia -- czyli podziale na dwie klasy (wygrana pierwszej lub drugiej drużyny) -- oraz klasyfikacji pokrycia lasu ze względu na dominujący gatunek drzew, spośród siedmiu możliwych gatunków, na podstawie informacji geograficznych o danym jego fragmencie. \\
Eksperymenty potwierdziły tezę o~bardzo szybkim działaniu sieci ELM (rzędu kilku sekund dla 1000 neuronów i~kilkudziesięciu tysięcy próbek danych), a~także ich satysfakcjonującej skuteczności -- przy czym wystąpiły zauważalne rozbieżności między aplikacją w~Pythonie i~Matlabie -- własna implementacja sieci osiągnęła znacznie lepsze rezultaty w~klasyfikacji pokrycia lasu.


\noindent \textbf{Słowa kluczowe:} big data, ekstremalne uczenie maszynowe, klasyfikacja, Matlab, Python, sieci neuronowe, uczenie maszynowe
\end{abstract}


\end{document}



















